\documentclass[a4paper,10pt]{article}

\usepackage[utf8]{inputenc}

\usepackage{mathtools}
\usepackage{amsfonts}
\usepackage{amssymb}
\usepackage{amsmath}
\usepackage{amsthm}
\usepackage{graphicx}
\usepackage{listings}
\usepackage{hyperref}
\usepackage[show]{ed}

\usepackage[english]{babel}

\title{Elements of Stochastic Processes\\Asssignment Sheet 2}
\author{Tom Wiesing}
\date{\today}

\begin{document}
\maketitle

\section{Exercise 5}
\subsection{Problem}

Assume that $X$ is a real valued continuous random variable with density $f_X$
and cumulative distribution function $F_X$. Show that $$
  \mathbb{E}\left[X\right] = \int_{-\infty}^{0}{F_x(t)\mathrm{d}t} + \int_{0}^{\infty}{1 - F_x(t)\mathrm{d}t}
$$.

\subsection{Solution}

For now assume that $X \geq 0$, i.e. $X$ never assumes negative values. Then we have:

$$
  \mathbb{E}\left[X\right] = \int_{0}^{\infty}{t f_x(t)\mathrm{d}t} = \int_{-\infty}^{\infty}{t P[X = t]\mathrm{d}t}
$$
$$
  = \int_{0}^{\infty}{P[X > t]\mathrm{d}t} = \int_{0}^{\infty}{1 - P[X \leq t]\mathrm{d}t} = \int_{0}^{\infty}{1 - F_X(t)\mathrm{d}t}
$$ Notice that in the middle of this computation, we switch the integral around.
Intuitively, the integral measures the area under the curve $t P[X = t]$ on the right hand side of the origin by slicing it vertically.
Equivalently, we can also measure the area by slicing it horizontally, which is what we are doing in the next line.

To compute the expected value of a generalised random variable (which may now be negative), we write it as a sum of two random variables $$
X = Y - Z
$$ with both $Y, Z \geq 0$. As the expected value is linear, we can then write

$$
  \mathbb{E}\left[X\right] = \mathbb{E}\left[Y\right] - \mathbb{E}\left[Z\right] = \int_{0}^{\infty}{1 - F_Y(t)\mathrm{d}t} - \int_{0}^{\infty}{1 - F_Z(t)\mathrm{d}t}
$$
$$
  = \int_{0}^{\infty}{1 - F_X(t)\mathrm{d}t} + \int_{-\infty}^{0}{F_x(t)\mathrm{d}t}
$$.

\section{Exercise 6}
\subsection{Problem}

For $\alpha \geq 1$ suppose that the random variable $X$ has the density function
$$
f_x(t) =
\begin{cases}
  \alpha e^{-\alpha t} & \mbox{if } t \geq 0 \\
  0 & \mbox{else}
\end{cases}
$$ Compute $\mathbb{E}[e^X]$.

\subsection{Solution}

$$
\mathbb{E}[e^X] = \int_{-\infty}^{\infty}{e^t \begin{cases}
  \alpha e^{-\alpha t} & \mbox{if } t \geq 0 \\
  0 & \mbox{else}
\end{cases}\mathrm{d}t}
= \int_{0}^{\infty}{e^t e^{-\alpha t} \mathrm{d}t}
= \int_{0}^{\infty}{e^{(1 - \alpha) t} \mathrm{d}t}
= \frac{1}{\alpha - 1}
$$.

{\raggedleft{}$\square$}

\section{Exercise 7}
\subsection{Problem}

Let $X$ and $Y$ be independent random variables with uniform distribution in the interval $[0, 1]$. Let $Z$ = $XY$. Find
\begin{enumerate}[a)]
  \item the joint probability distribution function of $X$ and $Y$,
  \item the joint probability distribution function of $X$ and $Z$,
  \item the joint density function of $X$ and $Z$
\end{enumerate}.

\subsection{Solution}
\subsubsection{a)}
$$
  F(x, y) = P[X \leq x, Y \leq y] = P[X \leq x]P[Y \leq y] = x y
$$

\subsubsection{b)}
$$
  F(x, z) = P[X \leq x, Z \leq z] = P[X \leq x, X + Y \leq z]
$$
$$
  = P[X \leq x, Y \leq z - x] = P[X \leq x]P[Y \leq z - x] = x (z - x)
$$

\subsubsection{c)}
$$
  f(x, z) = P[X = x, Z = z] = P[X = x, X + Y = z]
$$
$$
   = P[X = x, Y = z - x] = P[X = x]P[Y = z - x] = 1
$$

\section{Exercise 8}
\subsection{Problem}

Let $n \geq 2$ be a natural number and let the joint probability mass function of the discrete random variables $X$ and $Y$ be given by
$$
  p_{X,Y}(x, y) = \begin{cases}
    k (x + y) & \mbox{if }1 \leq x,y \leq n \\
    0 & \mbox{else}
  \end{cases}
$$

\begin{enumerate}[a)]
  \item Determine the value of the constant $k$.
  \item Determine the marginal probability mass functions of $X$ and $Y$.
  \item Find $P(X \geq Y)$. \\
  \textit{Hint: You can simplify the calculations by observing that $P(X \geq Y) = P(Y \geq X)$. }
\end{enumerate}

\subsection{Solution}
\subsubsection{a)}
We observe that $$
  1 = \sum_{x = 1}^{n}{\sum_{y = 1}^{n}{k (x + y)}} = k n^2 (n + 1)
$$ and thus have $k = \frac{1}{n^2 (n + 1)}$.
\subsubsection{b)}
We observe that $$
  P[X = x] = \sum_{y = 1}^{n}{k (x + y)} = k (\sum_{y = 1}^{n}{(x + y)})
  = k (nx + \sum_{y = 1}^{n}{y})
  = \frac{x}{n (n + 1)} + \frac{1}{2 n}
$$ By symmetry, we have
$$
  P[Y = y] = \frac{y}{n (n + 1)} + \frac{1}{2 n}
$$
\subsubsection{c)}
$$
  P[X \leq Y] = \sum_{y=1}^{n}{P[X \leq y, Y=y]} = \sum_{y=1}^{n}{\sum_{x=1}^{y}{P[X = x, Y = y]}}
$$
$$
  = \sum_{y=1}^{n}{\sum_{x=1}^{y}{k (x + y)}} = \frac{1}{2} k n (n + 1)^2 = \frac{1}{2n}
$$


\end{document}
